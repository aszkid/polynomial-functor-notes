% !TEX root=main.tex

\section{Note 1 -- preliminaries}

\subsection{Product and coproduct}

\begin{definition}[Product]
	Given a category $\calC$ and objects $X_1,X_2\in\calC$, we say that $X_1\times X_2$ is the \emph{product} of these two sets if there exist morphisms $\pi_1:X_1\times X_2 \to X_1$ and $\pi_2:X_1\times X_2\to X_2$, usually referred to as `projections', satisfying the following universal property: for any pair of morphisms $f_1:Y\to X_1$ and $f_2:Y\to X_2$, there is a unique morphism $f:Y\to X_1\times X_2$ such that $f_1=\pi_1 \circ f$ and $f_2=\pi_2 \circ f$, i.e. the following diagram commutes;

	\begin{figure}[!ht]
	\centering
	\begin{tikzcd}
	& \arrow[dl,"f_1"'] Y \arrow[d,dashed,"f"] \arrow[dr,"f_2"] &\\
	X_1 & \arrow[l,"\pi_1"] X_1\times X_2 \arrow[r,"\pi_2"'] & X_2
	\end{tikzcd}
	\end{figure}
\end{definition}

More intuitively, any pair of function $f_1,f_2$ into $X_1,X_2$ respectively factorize uniquely through the product; we can `naturally' pass through it, and then `project' back to our original codomain.

For $\calC=\cat{Set}$, the product manifests as the \emph{cartesian product}, with projection on the coordinates.

\begin{definition}[Coproduct]
	Given a category $\calC$ and objects $X_1,X_2\in\calC$, we say that $X_1\coprod X_2$ is the \emph{coproduct} of these two sets if there exist morphisms $i_1:X_1\to X_1\coprod X_2$ and $i_2:X_2\to X_1\coprod X_2$, usually referred to as `inclusions', satisfying the following universal property: for any pair of morphisms $f_1:X_1\to Y$ and $f_2:X_2\to Y$, there is a unique morphism $f:X_1\coprod X_2\to Y$ such that $f_1=f\circ i_1$ and $f_2=f\circ i_2$, i.e. the following diagram commutes;

	\begin{figure}[ht!]
	\centering
	\begin{tikzcd}
	& Y &\\
	X_1 \arrow[ur,"f_1"] \arrow[r,"i_1"'] & \arrow[u,dashed,"f"] X_1\coprod X_2 & \arrow[l,"i_2"] \arrow[ul,"f_2"'] X_2
	\end{tikzcd}
	\end{figure}
\end{definition}

This is of course the dual concept to that of the product; any pair of morphisms $f_1,f_2$ from $X_1,X_2$ respectively factorize uniquely through the coproduct; we can pass through it by means of inclusion, and then `naturally' end up in our original codomain.

For $\calC=\cat{Set}$, the coproduct manifests as the \emph{disjoint union}, with inclusion by identification.

Both concepts can be readily generalized to families of sets; we denote the categorical product by $\prod_{\alpha\in A} X_\alpha$ and the categorical coproduct by $\sum_{\alpha\in A} X_\alpha$.



\subsection{First encounter with polynomial functors}

After a quick exposure to the basic concepts of category theory (spread through a long year), we introduce the concept of a \emph{polynomial functor} as a categorification of the usual arithmetic polynomial. We start with the easy one-variable case:

\begin{definition}
	Given two sets $E,B$ and a function $p:E\to B$, we define the \emph{polynomial functor} $P:\cat{Set}\to\cat{Set}$ as follows:
    \begin{enumerate}
    \item For $X\in\cat{Set}$, we have $X \mapsto \sum_{b\in B} X^{E_b}$
    \item For $f:X\to Y$, we have $f(X) \mapsto \sum_{b\in B} f(X)^{E_b}$
    \end{enumerate}

    with $E_b=p\inv\{b\}$ the fibre over $b$.
\end{definition}

(Note that by $X^{E_b}$ we mean the set of all functions $E_b\to X$, essentially equivalent to $E_b$-indexed subsets of $X$.)

The following identifications are freely made, and easy to convince oneself of:
\begin{lemma}[\emph{Kock~1.1.1}]
	For $p:E\to B$ and $X\in\cat{Set}$, the following hold:
	\begin{itemize}
		\item $\sum_{b\in B} E_{b} = E$,
		\item $\sum_{b\in B} X = B\times X$,
		\item $\prod_{e\in E} X = X^E$.
		\end{itemize}
\end{lemma}


\subsection{Slices and coslices}

When extending the theory of single-variable polynomial functors to the multivariate case, we might consider functors of the form
\begin{align*}
	(\cat{Set},\cat{Set},\cat{Set}) &\to \cat{Set}\\
    (X_1,X_2,X_3) &\mapsto X_1^2X_2 + X_3.
\end{align*}

This is all good, but there is a nicer way to label variables: \emph{slice categories}. If we are working with a polynomial in three variables, when taking the fibers over each $b$, we would like to specify whether $Eb$ takes us to fibers belonging to $X_1$, $X_2$ or $X_3$. In other words, we want to consider a bigger set $X$ `sliced' in three different parts.

We can achieve exactly that by making an \emph{indexing set} $I=\{1,2,3\}$ and slicing $\cat{Set}$. The resulting category is $(\cat{Set}\downarrow I)$, defined as follows:
\begin{itemize}
\item \emph{Objects:} maps $\pi:X\to I$.
\item \emph{Morphisms:} functions $f:X\to Y$ such that the diagram
\end{itemize}

\begin{figure}[ht!]
\centering
\begin{tikzcd}
X \arrow[dr,"\pi_X"'] \arrow[rr,"f"] & & Y \arrow[dl,"\pi_Y"]\\
& I &
\end{tikzcd}
\end{figure}
commutes.

In more intuitive terms: we classify the elements in $X$ indexing them to $I$ in all possible ways, through `classifiers' $\pi_X$, and a map between two specific slices $(X,\pi_X)\to(Y,\pi_Y)$ only makes sense in the slice category if the indexing is preserved, by means of the commutative diagram. The elements of the slice category can be thought of as pairs $(X,\pi_X)$, but the function $\pi_X$ already encodes enough information.

And like most things in category theory, we have a dual concept to slices. Intuitively (hopefully); the opposite concept to `classification' is `specification'. Here comes the \emph{coslice category}: given an indexing set $I$, we coslice $\cat{Set}$ and get $(I\downarrow\cat{Set})$, defined as follows:
\begin{itemize}
\item \emph{Objects:} maps $\pi:I\to X$.
\item \emph{Morphisms:} functions $f:X\to Y$ such that the diagram
\end{itemize}

\begin{figure}[ht!]
\centering
\begin{tikzcd}
& I \arrow[dl,"\pi_X"'] \arrow[dr,"\pi_Y"] &\\
X \arrow[rr,"f"] & & Y
\end{tikzcd}
\end{figure}
commutes.

(explain `specifier' intuition ...)

\subsection{Pullbacks}
The categorical construction of the \emph{pullback} permeates thorough mathematics, and we use it widely. We define it here by its universal property, as is common in category theory.

\begin{definition}[Pullback]
	Given a category $\calC$, an object $Z$ and morphisms $f:X\to Z$ and $g:Y\to Z$, we call $X\times_{Z} Y$ the \emph{pullback} (or \emph{fibre product}) of $f$ and $g$ if (1) the following diagram commutes;

	\begin{center}
	\begin{tikzcd}
		X\times_Z Y \arrow[r,"p_2"] \arrow[d,"p_1"'] & Y \arrow[d,"g"]\\
		X \arrow[r,"f"'] & Z
	\end{tikzcd}
	\end{center}

	and (2) any other such commuting diagram factors uniquely through $X\times_Z Y$. More concretely, if we have some object $Q$ and morphisms $q_1:Q\to X$ and $q_2:Q\to Y$ making the diagram commute, then there is a unique morphism $u:Q\to X\times_Z Y$ such that $q_1 = p_1\circ u$ and $q_2 = p_2\circ u$, i.e. the following diagram commutes;

	\begin{center}
	\begin{tikzcd}
		Q \arrow[ddr,bend right,"q_1"'] \arrow[drr,bend left,"q_2"] \arrow[dr,dashed,"u"] & & \\
		& X\times_Z Y \arrow[r,"p_2"] \arrow[d,"p_1"'] & Y \arrow[d,"g"]\\
		& X \arrow[r,"f"'] & Z
	\end{tikzcd}
	\end{center}

\end{definition}

More intuitively; the pullback is the `most general way' to complete a pair of morphisms into a commutative square. In concrete, set-theoretical terms, the pullback organizes the agreeing fibres of two functions with equal codomain in the most efficient way possible.


\subsection{$p^*$ -- pullback functor}

The characteristic function $p:E\to B$ of a polynomial functor induces a functor $p^*:(\catset/ B)\to(\catset/ E)$ sending maps $f:Y\to B$ to the pullback $E\times_B Y$.

More concretely,
\begin{align*}
	p^* \left( Y_b \mid b\in B \right) &= E\times_B Y\\
		&= \sum_{e\in E} Y_{p(e)} = \left( Y_{p(e)} \mid e\in E \right)
\end{align*}
where maps are implicit. The fibre over some $e\in E$ is of course given by $Y_{p(e)}$. In more intuitive terms, this functor nicely organizes elements in $Y$ and $E$, with respect to fibres over $B$.

(From now on, we abuse notation by setting $\Hom_Q = \Hom_{\cat{Set}/Q}$.)


\subsection{Adjoint functors}

\begin{definition}[Adjoint, \emph{Leinster~2.1.1}]
	Whenever we have two functors
	\begin{tikzcd}
		\calA \arrow[r,shift left,"F"] & \arrow[l,shift left,"G"] \calB
	\end{tikzcd},
	we say that $F$ is \emph{left-adjoint} to $G$ and write $F \dashv G$ if
	\begin{align*}
		\Hom_\calB(F(A),B) \cong \Hom_\calA(F,G(B))
	\end{align*}
	naturally in $A\in\calA$ and $B\in\calB$.
\end{definition}

By \emph{naturally}, we mean that our choice of isomorphism is done in the most `general way'. Concretely, given $g\in\Hom_\calB(F(A),B)$, we denote by $\overline{g}\in\Hom_\calA(A,G(B))$ the map induced by the isomorphism of $\Hom$-sets, and in the other direction, $f\in\Hom_\calA(A,G(B))$ gives $\overline{f}\in\Hom_\calB(F(A),B)$. (Note that $\overline{\overline{f}}=f$.) Naturality then says the following;
\begin{align*}
	\overline{\left(F(A) \overset{g}{\longrightarrow} B \overset{q}{\longrightarrow} B'\right)} = \left( A \overset{\overline{g}}{\longrightarrow} G(B) \overset{G(q)}{\longrightarrow} G(B') \right)
\end{align*} ($\overline{q\circ g} = G(q)\circ\overline{g}$) for all $g$ and $q$, and
\begin{align*}
	\overline{\left( A' \overset{p}{\longrightarrow} A \overset{f}{\longrightarrow} G(B) \right)} = \left( F(A') \overset{F(p)}{\longrightarrow} F(A) \overset{\overline{f}}{\longrightarrow} B \right)
\end{align*}
for any $p$ and $q$.

For now, we care about two properties of adjoints, the first one empirical: (1) there is a good chance that a pair of opposing functors are adjoint, and (2) if a functor has a left-adjoint, it is unique up to isomorphism.


\subsubsection{Unit and counit}
Fixing $A\in\calA$, we have the identity morphism $F(A) \overset{1_\calB}{\longrightarrow} F(A)$, which by left-adjunction corresponds to a morphism $A \overset{\eta_A}{\longrightarrow} GF(A)$, letting $B=F(A)$ in the previous definition. The collection $\set{\eta_A}_{A\in\calA}$ defines the \emph{unit} of our adjunction, denoted by $\eta$.

Similarly, fixing $B\in\calB$, from the identity morphism $G(B) \overset{1_\calA}{\longrightarrow} G(B)$ we obtain by right-adjunction the morphism $FG(B) \overset{\epsilon_B}{\longrightarrow} B$, letting $A=G(B)$. Again, the collection $\set{\epsilon_B}_{B\in\calB}$ defines the \emph{counit} of our adjunction, denoted by $\epsilon$.

We claim that $\eta : 1_\calA \to G\circ F$ and $\epsilon : F\circ G \to 1_\calB$ are natural transformations. This should be intuitive, from the fact that how we obtained each unit and counit had nothing to do with specific $A$'s and $B$'s or morphisms in the respective categories: we only used the assumption that an identity exists for each element, and that we can adjoin it in a natural way (and this `natural' detail is key).

\begin{lemma}[Triangle identities, \emph{Leinster 2.2.2}]
	Given an adjunction $F \dashv G$ with unit $\eta$ and counit $\epsilon$, the triangles
	\begin{center}
	\begin{tikzcd}
		F \arrow[r,"F\circ\eta"] \arrow[dr,"1_F"'] & FGF \arrow[d,"\epsilon\circ F"]\\
		  & F
	\end{tikzcd}
	\quad
	\begin{tikzcd}
		G \arrow[r,"\eta\circ G"] \arrow[dr,"1_G"'] & GFG \arrow[d,"G\circ\epsilon"]\\
		  & G
	\end{tikzcd}
	\end{center}
	commute.
\end{lemma}
(Note that each arrow is a natural transformation, which play nicely with composition [\emph{Leinster, Remarks 1.3.24}].)
\begin{proof}
	First lets figure out where are these natural transformations coming from. We have the diagram
	\begin{center}
	\begin{tikzcd}
		\calB \arrow[r,bend left,"1_\calB"{name=I}] \arrow[r,bend right,"F\circ G"'{name=C}] & \calB \arrow[r,"G"] & \calA,
		\arrow[Rightarrow,from=C,to=I,"\epsilon"]
	\end{tikzcd}
	\end{center}
	which by horizontal composition gives
	\begin{center}
	\begin{tikzcd}
		\calB \arrow[r,bend left,"G"{name=G}] \arrow[r,bend right,"G\circ F\circ G"'{name=C}] & \calA
		\arrow[Rightarrow,from=C,to=G,"G\circ\epsilon"]
	\end{tikzcd},
	\end{center}
	with $G\circ\epsilon$ being the natural transformation we wanted.

	Similarly, the following diagram
	\begin{center}
	\begin{tikzcd}
		\calA \arrow[r,bend left,"1_\calA"{name=I}] \arrow[r,bend right,"G\circ F"'{name=C}] & \calA \arrow[r,"F"] & \calB
		\arrow[Rightarrow,from=I,to=C,"\eta"']
	\end{tikzcd}
	\end{center}
	gives by horizontal composition
	\begin{center}
	\begin{tikzcd}
		\calA \arrow[r,bend left,"F"{name=F}] \arrow[r,bend right,"F\circ G\circ F"'{name=C}] & \calB
		\arrow[Rightarrow,from=F,to=C,"F\circ\eta"']
	\end{tikzcd}.
	\end{center}

	All is left to check is that the triangles commute. We show this holds for this last triangle; the proof for the other one is dual. Fix $A\in\calA$, and consider the object $GF(A)$. By adjunction we get $\overline{1_{GF(A)}}=\epsilon_{F(A)}$, and by naturality we have
	\begin{align*}
		\overline{\left( A \overset{\eta_A}{\longrightarrow} GF(A) \overset{1_{GF(A)}}{\longrightarrow} GF(A) \right)} = F(A) \overset{F(\eta_A)}{\longrightarrow} FGF(A) \overset{\epsilon_{F(A)}}{\longrightarrow} FGF(A),
	\end{align*}
	but since the left-hand side simplifies to $\overline{\eta_A}=1_{F(A)}$, we get that $1_{F(A)}=\epsilon_{F(A)}\circ F(\eta_A)$ as desired.
\end{proof}

\subsection{$p_!$ -- left adjoint}
Given $p: E\to B$ and $X\overset{f}{\to} E$, we define the functor $p_! : (\cat{Set}/ E)\to(\cat{Set}/ B)$ by
\begin{align*}
	p_!\left( X_e \mid e\in E \right) &= \left( \sum_{e\in E_b} X_e \mid b\in B \right) = X.
\end{align*}

Note that $p_!(X)$ is a set over $B$ by simply post-composing $X\overset{f}{\to} E\overset{p}{\to} B$. We are simply organizing the fibres over $B$ instead of $E$.

\begin{proposition}[\emph{Kock 8.2.3}]
	$p_!$ is left adjoint to $p^*$.
\end{proposition}
\begin{proof}
	A morphism $u\in \Hom_B(p_!X,Y)$ makes the diagram
	\begin{center}
	\begin{tikzcd}
		X \arrow[rr,"u"] \arrow[dr,"p_!(f)"'] & & Y \arrow[dl]\\
			& B &
	\end{tikzcd},
	\end{center}
	commute, which by definition of $p_!$ is expanded into
	\begin{center}
	\begin{tikzcd}
		X \arrow[d,"f"'] \arrow[r,"u"] & Y \arrow[d]\\
		E \arrow[r,"p"'] & B
	\end{tikzcd}
	\end{center}
	(this follows from our observation on the previous paragraph). This is equivalent to giving a morphism $v\in\Hom_E(X,p^*Y)$ in the diagram
	\begin{center}
	\begin{tikzcd}
		X \arrow[dr,"v"] \arrow[ddr,bend right,"f"'] \arrow[drr,bend left,"u"] & &\\
		  & E\times_B Y \arrow[d] \arrow[r] & Y \arrow[d]\\
			& E \arrow[r,"p"'] & B
	\end{tikzcd},
	\end{center}
	which is unique by the universal property of the pullback.
\end{proof}


\subsubsection{Unit and counit of $p_! \dashv p^*$}

We wish to write explicitly the unit $\eta_X : X \to p^*p_!X$ of the adjunction $p_! \dashv p^*$ at $f:X\to E$. Note that
\begin{align*}
	(p^*p_!X)_e &= (p_!X)_{p(e)} = \left( \sum_{q\in E_{p(e)}} X_q \right),
\end{align*}

i.e. $p^*p_!\left( X_e \mid e\in E \right)= \left( \sum_{q\in E_{p(e)}} X_q \mid e\in E \right)$. The most natural way to define $\eta_X$ is by sending each $B$-fibre $X_e$ to the summand $X_q$ with $q=e$.

The counit $\epsilon_Y : p_!p^*Y \to Y$ at $g:Y\to B$ can be similarly understood:
\begin{align*}
	(p_!p^*Y)_b &= \left( \sum_{e\in E_b} (p^*Y)_e \right)\\
		&= \left( \sum_{e\in E_b} Y_{p(e)} \right)\\
		&= \left( \sum_{e\in E_b} Y_b \right) = E_b \times Y_b,
\end{align*}
i.e. $p_!p^*\left(Y_b \mid b\in B \right) = \left( E_b\times Y_b \mid b\in B \right)$ and so the most natural way to define $\epsilon_Y$ is by simply projecting down $(e,y) \mapsto y$.


\subsection{$p_*$ -- right adjoint}

Any function $p:E\to B$ also induces a functor $p_*: (\catset/E)\to(\catset/B)$. Given $X\overset{f}{\to} E$, we define
\begin{align*}
	p_*\left( X_e \mid e\in E \right) = \left( \prod_{e\in E_b} X_{e} \mid b\in B \right).
\end{align*}

This functor is synonymous with `multiplying the fibres'. We use to it multiply the monomials in the same term.

To specify an element in the fibre $(p_*X)_b$ is to give, for each $e\in E_b$, a subset $X_e\subset X$. In other words, it is equivalent to a collection of functions $\set{s:E_b\to X}\subset X^{E_b}$ such that the diagram
\begin{center}
\begin{tikzcd}
	  & X \arrow[d] \\
	E_b \arrow[ur,"s"] \arrow[r,"\subset" description] & E
\end{tikzcd}
\end{center}
commutes.

\begin{proposition}[\emph{Kock 8.2.10}]
	$p_*$ is right adjoint to $p^*$.
\end{proposition}
\begin{proof}
	Take any morhpism $u\in\Hom_E(p^*Y,X)$; this is equivalent to specifying, for each $e\in E$, a morphism $u_e: Y_{p(e)}\to X_e$. Hence
	\begin{align*}
		\Hom_E(p^*Y,X) &= \prod_{e\in E} X_e^{Y_{p(e)}}\\
			&= \prod_{b\in B} \prod_{e\in E_b} X_e^{Y_b}.
	\end{align*}

	Now take a morphism $v\in\Hom_B(Y,p_*X)$; this is equivalent to specifying, for each $b\in B$, a morphism $v_b: Y_b \to \prod_{e\in E_b} X_e$. Hence
	\begin{align*}
		\Hom_B(Y,p_*X) &= \prod_{b\in B} \left( \prod_{e\in E_b} X_e \right)^{Y_b} &&\text{justify next line!}\\
			&= \prod_{b\in B} \prod_{e\in E_b} X_e^{Y_b}.
	\end{align*}

	Therefore $\Hom_E(p^*Y, X) \cong \Hom_B(Y,p_*X)$. The fact that we have used no specific structure of $X$ and $Y$ should be argument enough that this isomorphism is natural.
\end{proof}

\subsubsection{Unit and counit of $p^* \dashv p_*$}

\pagebreak
