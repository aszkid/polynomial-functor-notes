% !TEX root=main.tex

\section{Day 1}

Course focus; \emph{finite-dimensional real and complex vector spaces}.

Why is linear algebra important in mathematics?
\begin{itemize}
\item \emph{Multivariate Calculus:} derivative of a function $f:\bbR^n\to\bbR^m$ at a point is a linear transformation.
\item \emph{Differential Geometry:} study of linear algebraic structures at every point of a smooth manifold (e.g; tangent and cotangent spaces).
\item \emph{Functional Analysis:} study of $\infty$-dimensional vector spaces (spaces of functions).
\item \emph{Abstract Algebra:} vector spaces are enriched groups.
\item \emph{Systems of ODEs:} non-linear systems are approximated by matrices.
\item \emph{Numerical Analysis:} ...
\end{itemize}

\separator

\subsection{Motivation for Vector Spaces}
\begin{enumerate}[Ex. A)]
\item Let $\bbR^n=\{n$-tuples $(x_1, \cdots, x_n)$ with $x_i\in\bbR\}$. $\bbR^n$ is closed under the usual operations of vector addition and scalar multiplication.

\item Let $n=3$, and consider $P=\{x\in\bbR^3 \mid 3x_1+2x_2+x_3=0\}$. This is a plane living in $\bbR^3$, crossing through the origin, defined by its normal vector $\vec{n}=(3,2,1)$. A vector $x\in\bbR^3$ belongs to $P$ if and only if $\vec{n}\cdot x = 0$. It is easy to check that vectors in $P$ are closed under addition and scalar multiplication.

	However, the plane defined by $P'=\{x\in\bbR^3 \mid 3x_1+2x_2+x_3=2\}$ (which does not cross through the origin) is not closed under those same operations. This is an example of an \emph{affine space}.

\item Consider the 2nd-order ODE given by $ay''(t)+by'(t)+cy(t)=0$ for some $a,b,c\in\bbR$. For any pair of solutions $y_1(t)$ and $y_2(t)$, the functions $(y_1+y_2)(t)$, $\alpha y_1(t)$ and $\alpha y_2(t)$ for some $\alpha\in\bbR$ are solutions as well.

	However, the equation $ay''(t)+by'(t)+cy(t)=f(t)$ for some non-zero function $f$ does not behave as nicely; solutions are not closed under addition and scalar multiplication.
\end{enumerate}

\subsection{Axioms of a Vector Space}
All examples above satisfy a precise set of conditions, which characterize \emph{vector spaces}. We focus at first on a specific type of vector space:

\begin{definition}
	A \emph{real vector space} is a set $V$ equipped with two operations (addition and scalar multiplication) satisfying the following axioms:
	\begin{enumerate}[{Ax.}1)]
		\item If $v,w\in V$, then $v+w=w+v$.
        \item If $u,v,w\in V$, then $u+(v+w)=(u+v)+w$.
        \item $\exists\vz\in V$ s.t. $v+\vz$ for $v\in V$. \label{ax:zerovec}
        \item If $v\in V$, $\exists -v\in V$ s.t. $v+(-v)=\vz$.
        \item $1v=v$ for $v\in V$.
        \item $(\alpha\beta)v=\alpha(\beta v)$ for $v\in V$ and $\alpha,\beta\in\bbR$.
        \item $\alpha(u+v)=\alpha u + \alpha v$ for $u,v\in V$ and $\alpha\in\bbR$.
        \item $(\alpha+\beta)v=\alpha v + \beta v$ for $v\in V$ and $\alpha,\beta\in\bbR$.
	\end{enumerate}
\end{definition}
\begin{remark*}
	By replacing $\bbR$ with $\bbC$ in the definition above, we get a \emph{complex vector space}.

    More generally, we can define a \emph{vector space over a field $\bbF$}, where the field $\bbF$ is a set endowed with two operations (addition and multiplication), behaving as we expect them to.
\end{remark*}

\subsection{Immediate Theorems}
We consider an arbitrary vector space $V$:
\begin{theorem}\label{thm:zerounique}
	$\vz\in V$ is unique.
\end{theorem}
\begin{theorem}
	Given $v\in V$, $-v$ is unique.
\end{theorem}
\begin{theorem}
	$0v=\vz$ and $-v=(-1)v$ for $v\in V$.
\end{theorem}
\begin{theorem}
	If $u+v=u+w$, then $v=w$ for $u,v,w\in V$.
\end{theorem}

We give the proof for Theorem \ref{thm:zerounique}:
\begin{proof}
	Assume that $\vz,\vz'\in V$ satisfy Axiom \ref{ax:zerovec}; i.e., $v+\vz=v$ and $v+\vz'=v$. By letting $v=\vz'$ in the first equation we get $\vz'+\vz=\vz'$, and by letting $v=\vz$ in the second equation we get $\vz+\vz'=\vz$. By commutativity we get that $\vz'+\vz=\vz'$ and $\vz'+\vz=\vz$, hence $\vz'=\vz$.
\end{proof}

\subsection{Some Vector Space examples}
\begin{enumerate}[Ex. A)]
\item For any field $\bbF$, let $\bbF^n=\{(x_1,\cdots,x_n) \mid x_i\in\bbF\}$ be endowed with usual vector addition and scalar multiplication (by scalars in $\bbF$). $\bbF^n$ is a vector space over $\bbF$.

\item Let $\calF(\bbR)=\{$real-valued functions on $\bbR\}$. By defining vector addition as $(f+g)(x)=f(x)+g(x)$ and scalar multiplication as $(\alpha f)(x)=\alpha f(x)$, we can show that $\calF(\bbR)$ is a vector space. Notice that the identity vector is $\vz(x)=0$ for all $x\in\bbR$, and that the additive inverse of any $f\in\calF(\bbR)$ is $(-f)(x)=-f(x)$.

\item Let $\calP(\bbF)=\{$polynomials with coefficients in $\bbF\}$. This is a vector space over $\bbF$.
\end{enumerate}

\subsection{Vector Subspaces}
\begin{definition}
	A \emph{subspace} $S$ of a vector space $V$ over a field $\bbF$ is a non-empty subset of $V$ satisfying the following conditions:
    \begin{enumerate}
    \item If $a,b\in S$, then $a+b\in S$.
    \item If $a\in S$ and $\lambda\in\bbF$, then $\lambda a\in S$.
    \end{enumerate}
\end{definition}

In other words; $S$ is a vector space by itself, with elements from $V$.

\begin{enumerate}[Ex. A)]
\item \emph{Line through the origin in $\bbR^2$:} ...
\item \emph{Plane through the origin in $\bbR^3$:} ...
\item \emph{Continuous functions on $\bbR$:} ...
\item \emph{Differentiable functions on $\bbR$:}
\end{enumerate}
