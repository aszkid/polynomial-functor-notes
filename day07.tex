% !TEX root=main.tex

\pagebreak
\section{Day 7}

\subsection{Rank-Nullity Theorem}
\begin{theorem}[Rank-Nullity]\label{thm:ranknull}
  $\dim V = \rank T + \nullity T$.
\end{theorem}

\emph{Example;} consider the linear transformation of projection on the $x$-axis, in $\bbR^2$. In matrix form (canonical basis), this is
\begin{align*}
A= \begin{bmatrix}
  1 & 0\\
  0 & 0
\end{bmatrix}.
\end{align*}

We compute both the rank and the nullity of this transformation. Consider $Av=(v_1,0)=v_1(1,0)$; so $\rank A = 1$. And for $Aw=0$, we need $w_1=0$, so $w=w_2(0,1)$; so $\nullity A = 1$. Hence $\dim\bbR^2=1+1=2$. Success!

\begin{lemma}
  $T$ is surjective iff $\rank T = \dim W$.
\end{lemma}
In other words; the rank measures close to surjectivity we are. Full rank implies surjectivity.
\begin{proof}
  $\Longrightarrow$ We have that $T(V)=W$, so $T$ is surjective.

  $\Longleftarrow$ We have that $\dim T(V)=\dim W$, and $T(V)$ is a subspace of $W$, so by Homework \#2, $T(V)=W$.
\end{proof}

\begin{lemma}
  $T$ is injective iff $\nullity T = 0$.
\end{lemma}
In other words; the nullity measures how close to injectivity we are. Full nullity (equal to zero) implies injectivity.
\begin{proof}
  $\Longleftarrow$ Consider two distinct $u,v\in V$. Hence $u-v\neq 0$, so $Tu-Tv\neq 0$. This implies that $Tu\neq Tv$, so $T$ is injective.

  $\Longrightarrow$ Contrapositive. Since $\nullity T > 0$, we have some non-zero $v\in n(T)$ such that $Tv=0$, but since $T(0)=0$ too, $T$ is not injective.
\end{proof}

And now for the proof of the \emph{Rank-Nullity Theorem} (Theorem \ref{thm:ranknull}):
\begin{proof}
  For a basis $\{v_1,\cdots,v_k\}$ of $n(T)$, we can extend it to a basis of the whole of $V$ by a previous theorem. We claim that $\{T(v_{k+1}), \cdots, T(v_{n})\}$.

  First, given $y\in T(V)$, then $y=Tx$ for some $x\in V$, so $x=\alpha_1 v_n + \cdots + \alpha_n v_n$. Hence
  \begin{align*}
    y = Tx &= T(\alpha_1 v_n + \cdots + \alpha_n v_n)\\
      &= (\alpha_1 Tv_1 + \cdots + \alpha_k Tv_k) + \cdots + \alpha_n Tv_n\\
      &= 0 + \alpha_{k+1} Tv_{k+1} + \cdots + \alpha_n Tv_n.
  \end{align*}
  So the set spans $T(V)$.

  We claim that the set is linearly independent... **
\end{proof}


Now follows a very useful theorem.
\begin{theorem}
  If $V$ is finite dimensional, and $T\in L(V,V)$, the following are equivalent:
  \begin{enumerate}[(1)]
    \item $T$ is invertible/bijective
    \item $T$ is injective
    \item $T$ is surjective
  \end{enumerate}
\end{theorem}
\begin{proof}
  $(1)\to(2)$ by definition.

  $(2)\to(3)$ ...

  $(3)\to(4)$ ...
\end{proof}




\subsection{Linear systems of equations -- intro}
