% !TEX root=main.tex

\section{Note 2 -- the polynomial functor}

\subsection{Multivariate polynomials}

We are ready to introduce polynomial functors in their fullest generality. Given a set-valued function $p:E\to B$, we define a few functors that, composed, will entirely characterize a polynomial functor. We want to multiple variables, and the standard strategy is to have an indexing function $i:E\to I$ so that each fibre $E_b$ is split into different labels.

To begin with, let $p^*:\slice{\catset}{B}\to\slice{\catset}{E}$ be the pullback functor defined in our previous note. We also let $p_!:\slice{\catset}{E}\to\slice{\catset}{B}$ be the functor which sends maps $f:X\to E$ to the composite $p_!(f):X\overset{f}{\to} E\overset{p}{\to} B$.

Our first task is to check that $p_!$ is left-adjoint to $p^*$; that is, for any $Y\overset{g}{\to} E\in\slice{\catset}{E}$ and $X\overset{f}{\to} B\in\slice{\catset}{B}$ there is a natural bijection
\[
	\hom_{\slice{\catset}{B}}(p_!g,f)\cong\hom_{\slice{\catset}{E}}(g,p^*f).
\]

\begin{proof}
	$\Longleftarrow$ Given any map $g\overset{\phi}{\to} p^*f$, we have the following commuting diagram:

    \begin{tikzcd}
    Y \arrow[dr,"g"] \arrow[rr,"\phi"] & & E\times_B X \arrow[dl,"p^*f"]\\
    & E &
    \end{tikzcd}

 	which can be naturally extended into


\end{proof}
