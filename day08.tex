% !TEX root=main.tex

\pagebreak
\section{Day 8}

\subsection{Gaussian elimination}

Valid \emph{row operations} are the following:
\begin{enumerate}[(1)]
  \item Exchange rows
  \item Scale rows
  \item Add a multiple of one row to another
\end{enumerate}

A \emph{pivot} is the first non-zero entry in a row of a given matrix.

\begin{definition}
  We say that a matrix is in \emph{row-echelon} form if each pivot is strictly to the right of the pivot in the previous row.
\end{definition}

The goal of \emph{Gaussian elimination} is to reduce the given matrix to row-echelon form.

** PROCESS

\begin{lemma}
  Row operations do not change the rank of a matrix.
\end{lemma}

\begin{definition}
  Two matrices $A$ and $B$ are row-equivalent if there exists a finite set of row operations that transform $A$ into $B$ and viceversa.
\end{definition}

We conclude from Gaussian elimination the following useful characterization:
\begin{itemize}
  \item \# pivots $= \rank(A)$.
  \item \# zero-rows $= \nullity(A)$.
\end{itemize}

So given any system of $m$ equations with $n$ unknowns, we form an $m\times n$ matrix so that $Ax=b$, and so

\begin{align*}
  \text{\emph{0 solutions}} &\Longleftrightarrow b\in \text{colspace}(A)\\
  \text{\emph{1 solution}} &\Longleftrightarrow \rank(A)=n\\
  \text{\emph{$\infty$ solutions}} &\Longleftrightarrow \nullity(A)=dim Sols??
\end{align*}
