% !TEX root=main.tex

\pagebreak
\section{Day 3}

\subsection{Towards the Important Theorem}
\begin{lemma}\label{lemma:dephelp}
  Let $S\subset V$ be a subspace; if $S$ is spanned by $n$ vectors $\{a_1, \cdots, a_n\}$, then for any set $\{b_1,\cdots,b_m\}\subset S$ with $m>n$ we have that $\{b_1,\cdots,b_m\}$ must be linearly dependent.
\end{lemma}
\begin{proof}
  Induction on $n$; for the base case $n=1$, it is clear that if $S=\sspan(v)$, for any set $\{b_1,\cdots,b_m\}\subset S$ we have that $\lambda_i v = b_i$. If some $\lambda_i = 0$ then $0\in S$ so $S$ is linearly dependent.

  If all $\lambda_i\neq 0$, let $u_i = \lambda_i v=a$, let $\mu_i = 1/\lambda_i$ for $1\leq i< m$ and $\mu_m = \frac{-(m-1)}{\lambda_m}$. Then we have
  \begin{align*}
    \mu_1 u_1 + \cdots + \mu_m u_m &= \mu_1 (\lambda_1 v) + \cdots + \mu_m (\lambda_m v)\\
      &= \frac{1}{\lambda_1}\lambda_1 v + \cdots + \frac{-(m-1)}{\lambda_m} \lambda_m v\\
      &= v((m-1)-(m-1))\\
      &= 0
  \end{align*}
  so $\{b_1,\cdots,b_m\}$ is linearly dependent.

  For the inductive step, assume that if some subspace $S$ is the span of $n-1$ vectors, then any subset of $m>n-1$ vectors is linearly independent. Consider now the case that $S$ is spanned by $n$ vectors, and consider any set $\{b_1,\cdots,b_m\}\subset S$ with $m>n(*****n-1??**)$. Since each $b_i\in\sspan(a_1,\cdots,a_n)$, then we can write each $b_i=\lambda_{i1} a_1 + \cdots + \lambda_{in} a_n$. We break this down in different cases:
  \begin{enumerate}[{Case} 1)]
    \item $\lambda_{i1} =0$ for all $i$. Then $b_i\sspan(a_2,\cdots,a_n)$, so $\{b_1,\cdots,b_m\}$ is linearly dependent by our inductive hypothesis.
    \item $\lambda_{i1} \neq 0$ for some $i$. Reorder our set such that $\lambda_{11}\neq 0$.

    \emph{Little trick;} choose $\alpha$ such that $b_2-\alpha_2 b_1 \in\sspan(a_2,\cdots,a_n)$, up to $b_m-\alpha_m b_1 \in\sspan(a_2,\cdots,a_n)$. If we let $\alpha_i = \frac{\lambda_{i1}}{\lambda_{11}}$, notice that
    \begin{align*}
      b_2-\frac{\lambda_{21}}{\lambda_{11}}b_1 &= (\lambda_{21} a_1 + \lambda_{2n} a_n) - \frac{\lambda_{21}}{\lambda_{11}}(\lambda_{11}a_1 + \cdots + \lambda_{1n} a_n)\\
        &= 0 + (\lambda_{22} - \frac{\lambda_{21}}{\lambda_{11}}\lambda_{12})a_2 + \cdots + (\lambda_{2n} - \frac{\lambda_{21}}{\lambda_{11}}\lambda_{1n}) a_n\\
        &= \lambda'_{22}a_2 + \cdots + \lambda'_{2n} a_n \in \sspan(a_2,\cdots,a_n)
    \end{align*}
    up to $b_n-\frac{\lambda_{m1}}{\lambda_{11}} b_1 = \lambda'_{m2}a_2 + \cdots + \lambda'_{mn}a_n\in\sspan(a_2,\cdots,a_n)$. So this set of differences is linearly dependent by our inductive hypothesis. Then there is some set of $\mu_i$, not all of which are equal to 0, such that
    \begin{align*}
      \mu_2(b_2 - \frac{\lambda_{21}}{\lambda_{11}}b_1) + \cdots + \mu_m (b_m - \frac{\lambda_{m1}}{\lambda_{11}}b_1) &= 0\\
      -(\cdots) b_1 + \mu_2 b_2 + \cdots + \mu_m b_m &= 0.
    \end{align*}
    Hence $\{b_1,\cdots,b_m\}$ is linearly dependent.
  \end{enumerate}
\end{proof}

We are now ready to prove our Important Theorem \ref{thm:basedep}.
\begin{proof}
  Assume $\{b_1,\cdots,b_m\}$ and $\{e_1,\cdots,e_n\}$ are basis for some vector space $V$, with $m\neq $n. WLOG, let $m>n$. Then by Lemma \ref{lemma:dephelp}, $\{b_1,\cdots,b_m\}$ must be linearly independent, a contradiction. Hence $m=n$.
\end{proof}


So by our theorem, the number $\dim(V)$ is well-defined if $V$ is finite-dimensional, and any set of $\dim(V)+k$ vectors of $V$ has to be linearly dependent.

\subsection{Finitely-generated vector spaces}

\begin{definition}
  A vector space $V$ is \emph{finitely-generated} if it is spanned by finitely many vectors.
\end{definition}

It is natural to ask the following:

\begin{theorem}
  Every finitely-generated vector space has a finite basis. Moreso, if $V=\sspan(a_1,\cdots,a_m)$, then $\{a_1,\cdots,a_m\}$ contains a basis.
\end{theorem}
\begin{proof}
  Induction; boring proof.
\end{proof}

In the other direction, it is clear that finite-dimensional vector spaces are finitely generated (by any of their bases); hence finite-dimensional $\Longleftrightarrow$ finitely-generated.

\begin{question}
  Does every vector space (including $\infty$-dimensional) have a basis?
\end{question}
\begin{answer}
  \emph{Yes}, assuming the axiom of choice.
\end{answer}


\begin{theorem}
  If $V$ is finite-dimensional and $\{b_1,\cdots,b_q\}\subset V$ are linearly independent, then there is a basis contained in this set.
\end{theorem}
\begin{proof}
  Induction backwards, boring.
\end{proof}



In general, to find the dimension of some vector space $V$, it sufficies to find a basis and count it.

\subsection{Constructing bigger spaces}
If $S$ and $T$ are subspaces of $V$, define $S+T:=\{s+t \mid s\in S, t\in T\}$.

\begin{theorem}
  If $S$ and $T$ are finite-dimensional, then $S\cap T$ and $S+T$ are finite-dimensional, and $\dim(S+T)+\dim(S\cap T) = \dim S + \dim T$.
\end{theorem}
\begin{proof}
  *****
\end{proof}
