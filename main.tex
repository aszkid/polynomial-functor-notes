\documentclass[12pt,oneside]{article}

\usepackage[margin=1in]{geometry}
\usepackage{amsmath,amsthm}
\usepackage{mathtools}
\usepackage{enumerate}
\usepackage[english]{babel}
\usepackage{csquotes}
\usepackage{fontspec}
\usepackage{unicode-math}
%\setmathfont[math-style=ISO,bold-style=ISO,vargreek-shape=TeX]{TeX Gyre Pagella Math}
\setmathfont{texgyrepagella-math.otf}
\setmainfont[Ligatures=TeX]{TeX Gyre Pagella}
\usepackage{tikz}
\usepackage{tikz-cd}
\usepackage[
	backend=biber,
	bibencoding=utf8,
	style=authoryear,
	citestyle=apa
]{biblatex}

% BIBLIO
\addbibresource{biblio.bib}

% MATH DEFINITIONS
% !TEX root=main.tex

\newenvironment{solution}
  {\begin{proof}[Solution]}
  {\end{proof}}

\newtheorem{theorem}{Theorem}
%[section]
%\newcounter{theorem}[section]
\newtheorem{proposition}[theorem]{Proposition}
\newtheorem*{lemma}{Lemma}
\newtheorem*{corollary}{Corollary}



\newtheorem{axiom}{Axiom}


\theoremstyle{definition}
\newtheorem{definition}[theorem]{Definition}
%\newtheorem{nondefinition}[theorem]{Non-Definition}
\newtheorem{exercise}[theorem]{Exercise}
\newtheorem{bonus}[theorem]{Bonus}
\newtheorem*{remark*}{Remark}
\newtheorem*{question}{Q}
%\newtheorem{warning}[theorem]{Warning}

\newenvironment{answer}
  {\begin{proof}[A]}
  {\end{proof}}

\DeclarePairedDelimiter\ceil{\lceil}{\rceil}
\DeclarePairedDelimiter\floor{\lfloor}{\rfloor}
\DeclarePairedDelimiter\parenth{\left(}{\right)}


\renewcommand{\_}[1]{\underline{ #1 }}


%% bold math capitals
\newcommand{\bA}{\mathbf{A}}
\newcommand{\bB}{\mathbf{B}}
\newcommand{\bC}{\mathbf{C}}
\newcommand{\bD}{\mathbf{D}}
\newcommand{\bE}{\mathbf{E}}
\newcommand{\bF}{\mathbf{F}}
\newcommand{\bG}{\mathbf{G}}
\newcommand{\bH}{\mathbf{H}}
\newcommand{\bI}{\mathbf{I}}
\newcommand{\bJ}{\mathbf{J}}
\newcommand{\bK}{\mathbf{K}}
\newcommand{\bL}{\mathbf{L}}
\newcommand{\bM}{\mathbf{M}}
\newcommand{\bN}{\mathbf{N}}
\newcommand{\bO}{\mathbf{O}}
\newcommand{\bP}{\mathbf{P}}
\newcommand{\bQ}{\mathbf{Q}}
\newcommand{\bR}{\mathbf{R}}
\newcommand{\bS}{\mathbf{S}}
\newcommand{\bT}{\mathbf{T}}
\newcommand{\bU}{\mathbf{U}}
\newcommand{\bV}{\mathbf{V}}
\newcommand{\bW}{\mathbf{W}}
\newcommand{\bX}{\mathbf{X}}
\newcommand{\bY}{\mathbf{Y}}
\newcommand{\bZ}{\mathbf{Z}}

%% blackboard bold math capitals
\newcommand{\bbA}{\mathbb{A}}
\newcommand{\bbB}{\mathbb{B}}
\newcommand{\bbC}{\mathbb{C}}
\newcommand{\bbD}{\mathbb{D}}
\newcommand{\bbE}{\mathbb{E}}
\newcommand{\bbF}{\mathbb{F}}
\newcommand{\bbG}{\mathbb{G}}
\newcommand{\bbH}{\mathbb{H}}
\newcommand{\bbI}{\mathbb{I}}
\newcommand{\bbJ}{\mathbb{J}}
\newcommand{\bbK}{\mathbb{K}}
\newcommand{\bbL}{\mathbb{L}}
\newcommand{\bbM}{\mathbb{M}}
\newcommand{\bbN}{\mathbb{N}}
\newcommand{\bbO}{\mathbb{O}}
\newcommand{\bbP}{\mathbb{P}}
\newcommand{\bbQ}{\mathbb{Q}}
\newcommand{\bbR}{\mathbb{R}}
\newcommand{\bbS}{\mathbb{S}}
\newcommand{\bbT}{\mathbb{T}}
\newcommand{\bbU}{\mathbb{U}}
\newcommand{\bbV}{\mathbb{V}}
\newcommand{\bbW}{\mathbb{W}}
\newcommand{\bbX}{\mathbb{X}}
\newcommand{\bbY}{\mathbb{Y}}
\newcommand{\bbZ}{\mathbb{Z}}

%% script math capitals
\newcommand{\sA}{\mathscr{A}}
\newcommand{\sB}{\mathscr{B}}
\newcommand{\sC}{\mathscr{C}}
\newcommand{\sD}{\mathscr{D}}
\newcommand{\sE}{\mathscr{E}}
\newcommand{\sF}{\mathscr{F}}
\newcommand{\sG}{\mathscr{G}}
\newcommand{\sH}{\mathscr{H}}
\newcommand{\sI}{\mathscr{I}}
\newcommand{\sJ}{\mathscr{J}}
\newcommand{\sK}{\mathscr{K}}
\newcommand{\sL}{\mathscr{L}}
\newcommand{\sM}{\mathscr{M}}
\newcommand{\sN}{\mathscr{N}}
\newcommand{\sO}{\mathscr{O}}
\newcommand{\sP}{\mathscr{P}}
\newcommand{\sQ}{\mathscr{Q}}
\newcommand{\sR}{\mathscr{R}}
\newcommand{\sS}{\mathscr{S}}
\newcommand{\sT}{\mathscr{T}}
\newcommand{\sU}{\mathscr{U}}
\newcommand{\sV}{\mathscr{V}}
\newcommand{\sW}{\mathscr{W}}
\newcommand{\sX}{\mathscr{X}}
\newcommand{\sY}{\mathscr{Y}}
\newcommand{\sZ}{\mathscr{Z}}

\newcommand{\calG}{\mathcal{G}}
\newcommand{\calT}{\mathcal{T}}
\newcommand{\calR}{\mathcal{R}}
\newcommand{\calV}{\mathcal{V}}
\newcommand{\calE}{\mathcal{E}}
\newcommand{\calW}{\mathcal{W}}
\newcommand{\calU}{\mathcal{U}}
\newcommand{\calF}{\mathcal{F}}
\newcommand{\calP}{\mathcal{P}}

\providecommand{\ar}{\rightarrow}
\providecommand{\arr}{\longrightarrow}

\newcommand{\RHS}{\text{RHS}}
\newcommand{\LHS}{\text{LHS}}

\newcommand{\bfz}{\mathbf{0}}

\providecommand{\inv}{^{-1}}

\DeclareMathOperator{\LP}{LP}
\DeclareMathOperator{\im}{Im}
\DeclareMathOperator{\inte}{int}

% VECTOR STUFF
\newcommand{\vz}{\mathbf{0}}
\DeclareMathOperator{\sspan}{span}

\DeclareMathOperator{\rank}{rank}
\DeclareMathOperator{\nullity}{nullity}
\DeclareMathOperator{\colspace}{ColumnSpace}
\DeclareMathOperator{\area}{Area}

\DeclarePairedDelimiterX{\inp}[2]{\langle}{\rangle}{#1, #2}
\DeclarePairedDelimiterX{\norm}[1]{\lVert}{\rVert}{#1}


% category stuff
\newcommand{\cat}{%
	\mathbf%
}
\newcommand{\catSet}{\cat{Set}}

\newcommand{\slice}[2]{(#1\downarrow #2)}

% HELPERS
\newcommand{\separator}
{
	\vspace{.2in}
    \centerline{--------------------------------}
    \vspace{.2in}
}

\begin{document}

% TITLEPAGE
\title{
	A mathematical promenade\\\vspace{0.15in}
    \large{Notes on category theory and polynomial functors}
}
\author{Pol Gómez Riquelme \\ \texttt{gomezp@uchicago.edu}}
\date{June--August, 2017}
\maketitle
\tableofcontents
\pagebreak

% MAIN CONTENTS
% !TEX root=main.tex

\section{Note 1}

\subsection{Product and coproduct}

\begin{definition}[Product]
	Given a category $\calC$ and objects $X_1,X_2\in\calC$, we say that $X_1\times X_2$ is the \emph{product} of these two sets if there exist morphisms $\pi_1:X_1\times X_2 \to X_1$ and $\pi_2:X_1\times X_2\to X_2$, usually referred to as `projections', satisfying the following universal property: for any pair of morphisms $f_1:Y\to X_1$ and $f_2:Y\to X_2$, there is a unique morphism $f:Y\to X_1\times X_2$ such that $f_1=\pi_1 \circ f$ and $f_2=\pi_2 \circ f$, i.e. the following diagram commutes;

	\begin{figure}[!ht]
	\centering
	\begin{tikzcd}
	& \arrow[dl,"f_1"'] Y \arrow[d,dashed,"f"] \arrow[dr,"f_2"] &\\
	X_1 & \arrow[l,"\pi_1"] X_1\times X_2 \arrow[r,"\pi_2"'] & X_2
	\end{tikzcd}
	\end{figure}
\end{definition}

More intuitively, any pair of function $f_1,f_2$ into $X_1,X_2$ respectively factorize uniquely through the product; we can `naturally' pass through it, and then `project' back to our original codomain.

For $\calC=\cat{Set}$, the product manifests as the \emph{cartesian product}, with projection on the coordinates.

\begin{definition}[Coproduct]
	Given a category $\calC$ and objects $X_1,X_2\in\calC$, we say that $X_1\coprod X_2$ is the \emph{coproduct} of these two sets if there exist morphisms $i_1:X_1\to X_1\coprod X_2$ and $i_2:X_2\to X_1\coprod X_2$, usually referred to as `inclusions', satisfying the following universal property: for any pair of morphisms $f_1:X_1\to Y$ and $f_2:X_2\to Y$, there is a unique morphism $f:X_1\coprod X_2\to Y$ such that $f_1=f\circ i_1$ and $f_2=f\circ i_2$, i.e. the following diagram commutes;

	\begin{figure}[ht!]
	\centering
	\begin{tikzcd}
	& Y &\\
	X_1 \arrow[ur,"f_1"] \arrow[r,"i_1"'] & \arrow[u,dashed,"f"] X_1\coprod X_2 & \arrow[l,"i_2"] \arrow[ul,"f_2"'] X_2
	\end{tikzcd}
	\end{figure}
\end{definition}

This is of course the dual concept to that of the product; any pair of morphisms $f_1,f_2$ from $X_1,X_2$ respectively factorize uniquely through the coproduct; we can pass through it by means of inclusion, and then `naturally' end up in our original codomain.

For $\calC=\cat{Set}$, the coproduct manifests as the \emph{disjoint union}, with inclusion by identification.

Both concepts can be readily generalized to families of sets; we denote the categorical product by $\prod_{\alpha\in A} X_\alpha$ and the categorical coproduct by $\sum_{\alpha\in A} X_\alpha$.



\subsection{First encounter with polynomial functors}

After a quick exposure to the basic concepts of category theory (spread through a long year), we introduce the concept of a \emph{polynomial functor} as a categorification of the usual arithmetic polynomial. We start with the easy one-variable case:

\begin{definition}
	Given two sets $E,B$ and a function $p:E\to B$, we define the \emph{polynomial functor} $P:\cat{Set}\to\cat{Set}$ as follows:
    \begin{enumerate}
    \item For $X\in\cat{Set}$, we have $X \mapsto \sum_{b\in B} X^{E_b}$
    \item For $f:X\to Y$, we have $f(X) \mapsto \sum_{b\in B} f(X)^{E_b}$
    \end{enumerate}

    with $E_b=p\inv\{b\}$ the fibre over $b$.
\end{definition}

(Note that by $X^{E_b}$ we mean the set of all functions $E_b\to X$, essentially equivalent to $E_b$-indexed subsets of $X$.)

The following identifications are freely made, and easy to convince oneself of:
\begin{lemma}[\emph{Kock~1.1.1}]
	For $p:E\to B$ and $X\in\cat{Set}$, the following hold:
	\begin{itemize}
		\item $\sum_{b\in B} E_{b} = E$,
		\item $\sum_{b\in B} X = B\times X$,
		\item $\prod_{e\in E} X = X^E$.
		\end{itemize}
\end{lemma}


\subsection{Slices and coslices}

When extending the theory of single-variable polynomial functors to the multivariate case, we might consider functors of the form
\begin{align*}
	(\cat{Set},\cat{Set},\cat{Set}) &\to \cat{Set}\\
    (X_1,X_2,X_3) &\mapsto X_1^2X_2 + X_3.
\end{align*}

This is all good, but there is a nicer way to label variables: \emph{slice categories}. If we are working with a polynomial in three variables, when taking the fibers over each $b$, we would like to specify whether $Eb$ takes us to fibers belonging to $X_1$, $X_2$ or $X_3$. In other words, we want to consider a bigger set $X$ `sliced' in three different parts.

We can achieve exactly that by making an \emph{indexing set} $I=\{1,2,3\}$ and slicing $\cat{Set}$. The resulting category is $(\cat{Set}\downarrow I)$, defined as follows:
\begin{itemize}
\item \emph{Objects:} maps $\pi:X\to I$.
\item \emph{Morphisms:} functions $f:X\to Y$ such that the diagram
\end{itemize}

\begin{figure}[ht!]
\centering
\begin{tikzcd}
X \arrow[dr,"\pi_X"'] \arrow[rr,"f"] & & Y \arrow[dl,"\pi_Y"]\\
& I &
\end{tikzcd}
\end{figure}
commutes.

In more intuitive terms: we classify the elements in $X$ indexing them to $I$ in all possible ways, through `classifiers' $\pi_X$, and a map between two specific slices $(X,\pi_X)\to(Y,\pi_Y)$ only makes sense in the slice category if the indexing is preserved, by means of the commutative diagram. The elements of the slice category can be thought of as pairs $(X,\pi_X)$, but the function $\pi_X$ already encodes enough information.

And like most things in category theory, we have a dual concept to slices. Intuitively (hopefully); the opposite concept to `classification' is `specification'. Here comes the \emph{coslice category}: given an indexing set $I$, we coslice $\cat{Set}$ and get $(I\downarrow\cat{Set})$, defined as follows:
\begin{itemize}
\item \emph{Objects:} maps $\pi:I\to X$.
\item \emph{Morphisms:} functions $f:X\to Y$ such that the diagram
\end{itemize}

\begin{figure}[ht!]
\centering
\begin{tikzcd}
& I \arrow[dl,"\pi_X"'] \arrow[dr,"\pi_Y"] &\\
X \arrow[rr,"f"] & & Y
\end{tikzcd}
\end{figure}
commutes.

(explain `specifier' intuition ...)

\subsection{Pullbacks}
The categorical construction of the \emph{pullback} permeates thorough mathematics, and we use it widely. We define it here by its universal property, as is common in category theory.

\begin{definition}[Pullback]
	Given a category $\calC$, an object $Z$ and morphisms $f:X\to Z$ and $g:Y\to Z$, we call $X\times_{Z} Y$ the \emph{pullback} (or \emph{fibre product}) of $f$ and $g$ if (1) the following diagram commutes;

	\begin{center}
	\begin{tikzcd}
		X\times_Z Y \arrow[r,"p_2"] \arrow[d,"p_1"'] & Y \arrow[d,"g"]\\
		X \arrow[r,"f"'] & Z
	\end{tikzcd}
	\end{center}

	and (2) any other such commuting diagram factors uniquely through $X\times_Z Y$. More concretely, if we have some object $Q$ and morphisms $q_1:Q\to X$ and $q_2:Q\to Y$ making the diagram commute, then there is a unique morphism $u:Q\to X\times_Z Y$ such that $q_1 = p_1\circ u$ and $q_2 = p_2\circ u$, i.e. the following diagram commutes;

	\begin{center}
	\begin{tikzcd}
		Q \arrow[ddr,bend right,"q_1"'] \arrow[drr,bend left,"q_2"] \arrow[dr,dashed,"u"] & & \\
		& X\times_Z Y \arrow[r,"p_2"] \arrow[d,"p_1"'] & Y \arrow[d,"g"]\\
		& X \arrow[r,"f"'] & Z
	\end{tikzcd}
	\end{center}

\end{definition}

More intuitively; the pullback is the `most general way' to complete a pair of morphisms into a commutative square. In concrete, set-theoretical terms, the pullback organizes the agreeing fibres of two functions with equal codomain in the most efficient way possible.


\subsection{$p^*$ -- pullback functor}

The characteristic function $p:E\to B$ of a polynomial functor induces a functor $p^*:(\catset\downarrow B)\to(\catset\downarrow E)$ sending maps $f:Y\to B$ to the pullback $Y\times_B E$.

More concretely, $p^* Y = Y\times_B E = \sum_{e\in E} Y_{p(e)}$ where maps are implicit. The fibre over some $e\in E$ is of course given by $Y_{p(e)}$. In more intuitive terms, this functor nicely organizes elements in $Y$ and $E$, with respect to fibres over $B$.


\subsection{Adjoint functors}

\begin{definition}[Adjoint, \emph{Leinster~2.1.1}]
	Whenever we have two functors
	\begin{tikzcd}
		\calA \arrow[r,shift left,"F"] & \arrow[l,shift left,"G"] \calB
	\end{tikzcd},
	we say that $F$ is \emph{left-adjoint} to $G$ and write $F \dashv G$ if
	\begin{align*}
		\Hom_\calB(F(A),B) \cong \Hom_\calA(F,G(B))
	\end{align*}
	naturally in $A\in\calA$ and $B\in\calB$.
\end{definition}

By \emph{naturally}, we mean that our choice of isomorphism is done in the most `general way'. Concretely, given $g\in\Hom_\calB(F(A),B)$, we denote by $\overline{g}\in\Hom_\calA(A,G(B))$ the map induced by the isomorphism of $\Hom$-sets, and in the other direction, $f\in\Hom_\calA(A,G(B))$ gives $\overline{f}\in\Hom_\calB(F(A),B)$. (Note that $\overline{\overline{f}}=f$.) Naturality then says the following;
\begin{align*}
	\overline{\left(F(A) \overset{g}{\longrightarrow} B \overset{q}{\longrightarrow} B'\right)} = \left( A \overset{\overline{g}}{\longrightarrow} G(B) \overset{G(q)}{\longrightarrow} G(B') \right)
\end{align*} ($\overline{q\circ g} = G(q)\circ\overline{g}$) for all $g$ and $q$, and
\begin{align*}
	\overline{\left( A' \overset{p}{\longrightarrow} A \overset{f}{\longrightarrow} G(B) \right)} = \left( F(A') \overset{F(p)}{\longrightarrow} F(A) \overset{\overline{f}}{\longrightarrow} B \right)
\end{align*}
for any $p$ and $q$.

For now, we care about two properties of adjoints, the first one empirical: (1) there is a good chance that a pair of opposing functors are adjoint, and (2) if a functor has a left-adjoint, it is unique up to isomorphism.


\subsection{Adjoint functors through units and counits}
Fixing $A\in\calA$, we have the identity morphism $F(A) \overset{1_\calB}{\longrightarrow} F(A)$, which by left-adjunction corresponds to a morphism $A \overset{\eta_A}{\longrightarrow} GF(A)$, letting $B=F(A)$ in the previous definition. This gives the collection of \emph{units} of our adjunction (note that each $\eta_{(-)}$ is the adjoint of the identity morphism in $\calB$).

Similarly, fixing $B\in\calB$, from the identity morphism $G(B) \overset{1_\calA}{\longrightarrow} G(B)$ we obtain by right-adjunction the morphism $FG(B) \overset{\epsilon_B}{\longrightarrow} B$, letting $A=G(B)$. This again gives the collection of \emph{counits} of our adjunction (with each $\epsilon_{(-)}$ being the adjoint of the identity morphism in $\calA$).

We claim that $\eta : 1_\calB \to G\circ F$ and $\epsilon : F\circ G \to 1_\calA$ are natural transformations. This should be intuitive, from the fact that how we obtained each unit and counit had nothing to do with specific $A$'s and $B$'s or morphisms in the respective categories: we only used the assumption that an identity exists for each element, and that we can adjoin it in a natural way (and this `natural' detail is key).

\subsection{$p_*$ -- adjoint functor}

\subsection{$p_!$ -- ?? functor}

\subsection{Adjoints}
The idea of two morphisms being adjoint arises everywhere in mathematics (i bet you hadn't heard that before). Let's see an example: consider the forgetful functor $U:\cat{Vect_k}\to\cat{Set}$ that sends a vector space $V$ to its underlying set $U(V)$, forgetting the structure, and the functor $F:\cat{Set}\to\cat{Vect_k}$ which constructs the free vector space from any set $S$. Instead of explicitly writing $F$, it is most convenient to consider a simple case and then state the universal property which makes the general idea of a \emph{free functor} useful.

For a set $S=\{a,b,c\}$, we define the \emph{free monoid} $F(S)$ to be all strings of `words' from $S$, e.g. $abc\in F(S)$, $aaaaaa\in F(S)$, etc. with concatenation. Furthermore, for $F$ to be a functor we must make sure that functions $S\to S'$ are nicely mapped to monoid homomorphisms $F(S)\to F(S')$ ; ....??? Note that most monoids can be constructed from the free functor on some set, but that $F\circ U\neq 1$.

% !TEX root=main.tex

\section{Note 2 -- the polynomial functor}

\subsection{Multivariate polynomials}

In its full generality, a polynomial functor $P:(\catset/I)\to(\catset/J)$ is described by a diagram of the form
\begin{center}
\begin{tikzcd}
    & E \arrow[dl,"s"'] \arrow[r,"p"] & B \arrow[dr,"t"] & \\
  I &   &   & J\\
    & X \arrow[ul,dashed] &   &
\end{tikzcd}
\end{center}

and it is explicitly given by $P=t_!p_*s^*$. To see how this makes sense, we unpack the definition:
\begin{align*}
  P(X) &= t_!p_*s^*\left( X_i \mid i\in I \right)\\
    &= t_!p_*\left( X_{s(e)} \mid e\in E \right)\\
    &= t_!\left( \prod_{e\in E_b} X_{s(e)} \mid b\in B \right)\\
    &= \left( \sum_{b\in B_j}\prod_{e\in E_b} X_{s(e)} \mid j\in J \right).
\end{align*}

Note that we can coherently follow the chain of functors
\[
  \catset/I \overset{s^*}{\rightarrow} \catset/E \overset{p_*}{\rightarrow} \catset/B \overset{t_!}{\rightarrow} \catset/J
\]
by looking at each line in the expansion. First, we group fibres of $X$ over $s(e)$ to agree with $E$ in terms of the indexing set; then we multiply the fibres that get mapped to the same term in the polynomial, represented by each element in $B$; and finally we add all the terms together and produce the polynomials indexed by $J$.

\subsection{Example}

An example might help clarify this abstract talk into more concrete terms. Consider the polynomial functor $X_1+X_1X_2 + X_1^2X_3$: this is a polynomial functor in three variables, so $I=\set{1,2,3}$; it has three terms, so $B=\set{a,b,c}$; it is made up of a total of six monomials, so $E=\set{(a,1), (b,1), (b,2), (c,1), (c,1)', (c,3)}$. In this case, $J=\set{\alpha}$ gives a single polynomial.

The map $s$ projects the second element of each pair; the map $p$ projects the first one; and $t$ sends everything to $\alpha$. See Table~\ref{table:polyrep} for a succint view.
\begin{table}[]
\centering
\caption{Graphical representation of $X_1+X_1X_2+X_1^2X_3$. The left column is $I$, the center region is $E$, and the bottom row is $B$. `Accumulate' points vertically ($p$) or horizontally ($s$).}
\begin{tabular}{c|ccc}\label{table:polyrep}
1 & $\bullet$ & $\bullet$ & $\bullet\bullet$\\
2 &   & $\bullet$ &   \\
3 &   &   & $\bullet$ \\ \hline
  & a  & b  & c
\end{tabular}
\end{table}


\pagebreak

% !TEX root=main.tex

\pagebreak

\section{Note 3 -- symmetry}

\subsection{Characterization of symmetric functors}

Given a symmetric functor $P$ defined by $I \overset{s}{\gets} E \overset{p}{\to} B$, it is reasonable to say that $P$ is \emph{symmetric} if given any $I$-permutation $\sigma:I\to I$, we have an isomorphic natural transformation $P \cong P\circ \sigma^*$.

\subsection{When sometimes sets are not enough}

We begin the study of symmetric functors by considering this very simple symmetric polynomial:
\[
P(X_1,X_2,X_3) = X_1X_2 + X_1X_3 + X_2X_3.
\]

Let $I \overset{s}{\gets} E \overset{p}{\to} B$ be sets and set-valued functions representing the above polynomial functor $P$. We might expect a nice combinatorial description of $E$, $B$ and their corresponding maps $s$ and $p$. The following picture gives a hint:

(picture)

In other words, $B$ is the set of all 2-element subsets of $I=\{1,2,3\}$, giving $\binom{3}{2}=3=|B|$ total monomials. To recover $E$, we essentially `repeat' these subsets with a chosen point, which denotes the labeling; $E$ is the the of pointed 2-element subsets of $I$. Summing up, we have

\begin{align*}
I &= \{1,2,3\}\\
B &= \{\{1,2\},\{1,3\},\{2,3\}\}\\
E &= \{\{\underline{1},2\}, \{1,\underline{2}\}, \{\underline{1},3\}, \{1,\underline{3}\},\{\underline{2},3\}, \{2,\underline{3}\}\}
\end{align*}

with maps $s:E\to I$ simply returning the chosen point, and $p:E\to B$ returning the original 2-subset.

To see how $P$ is indeed symmetric, we fix a 3-permutation $\sigma\in\mathfrak{S}_3$ and (...)

Now consider the following slightly more complicated symmetric polynomial:
\[
P(X_1,X_2,X_3) = X_1X_2 + X_1X_3 + X_2X_3 + X_1^2 + X_2^2 + X_3^2.
\]

We desire a similar combinatorial description; $B$ could be the set of all 2-subsets of $I=\{1,2,3\}$ with repetitions allowed, or rather, the set of all 2-tuples of $I$. This is the same thing as the set of all functions from $\underline{2}=\{1,2\}$ to $I$. But careful! Now we have \emph{too much} stuff; points $(1,2)$ and $(2,1)$ produce monomials $X_1X_2$ and $X_2X_1$, which we of course identify by commutativity. Thus, we need to equip $B$ with an equivalence relation that collapses these points.

Let $B=\{\underline{2}\to I\}/\sim$, with $f\sim g$ if given any permutation $\sigma\in\mathfrak{S}_2$, the diagram

\begin{figure}[!ht]
\centering
\begin{tikzcd}[row sep=tiny]
\underline{2}  \arrow[dr,"f"] \arrow[dd,"\sigma"] & \\
 & I\\
\underline{2} \arrow[ur,"g"] &
\end{tikzcd}
\end{figure}

commutes. Now we have $|B|=\binom{3+2-1}{2}=6$, which are exactly the terms of $P$. To recover $E$, we follow the same strategy; we copy $B$ choosing a point for every subset. In other words, we let $E=\{\underline{1}\to\underline{2}\to I\}$, but the need to remove repetitions arises again (we consider pairs $(\underline{1},2)$ and $(2,\underline{1})$ to be equivalent!), so it is a most reasonable choice to apply the quotient through the equivalence $\approx$, with pairs $(a,f)\approx(b,g)$ if for any 2-permutation $\sigma$, the diagram

\begin{figure}[!ht]
\centering
\begin{tikzcd}[row sep=tiny]
 & \underline{2} \arrow[dr,"f"] \arrow[dd,"\sigma"] & \\
\underline{1} \arrow[ur,"a"] \arrow[dr,"b"] & & I\\
 & \underline{2} \arrow[ur,"g"] &
\end{tikzcd}
\end{figure}

commutes. Now the map $s:E\to I$ is defined by the evaluation $s(\underline{1}\overset{a}{\to}\underline{2}\overset{f}{\to} I) = f\circ a(1)$, and $p:E\to B$ simply returns $f$ and applies the equivalence relation $\sim$.

This looks good, but let us check whether simple cardinality conditions on the fibres are satisfied by this setup. Taking $(1,2)$ as the representative of its equivalence class in $B/\sim$, we expect $|p\inv((1,2))|=2$; indeed, the fibre is $\{(\underline{1},2),(1,\underline{2})\}$. Then the polynomial functor evaluates at that term with $X_1X_2$, just as expected! Now take the fibre of $(1,1)$; we expect $|p\inv((1,1))|=2$, but notice what happens: the term $(\underline{1},1)$ gets identified with $(1,\underline{1})$, and so we end up with $|p\inv((1,1))|=1$, which leaves us with the monomial $X_1$ of degree one!

It has everything to do with the action of $\mathfrak{S}_2$ on $E$ \emph{not being free}. Consider the pair $(1,1)$, and two \emph{distinct} endomorphisms $Id$ and $\sigma$ (actions of $\mathfrak{S}_2$), with $Id$ taking the each coordinate to itself, and $\sigma$ exchanging them. In set-theoretic terms, the morphisms $Id$ and $\sigma$ are identified, because their set-theoretic evaluation is the same, and set-valued functions do not preserve that much structure. If we enrich our framework and preserve the fact that $Id\neq\sigma$, we might be able to expect a map that takes this detail into account. Here enter groupoids.

\subsubsection{Groupoids are sets all right -- but better}

\begin{figure}[!ht]
\centering
\begin{tikzcd}[column sep=tiny]
B =& (2,3) \arrow[rrrrrrrr,bend left,leftrightarrow,"\sigma"] \arrow[loop below,"Id"] & (1,3) \arrow[rrrrrr,bend left,leftrightarrow,"\sigma"] \arrow[loop below,"Id"] & (1,2) \arrow[rrrr,bend left,leftrightarrow,"\sigma"] \arrow[loop below,"Id"] & (1,1) \arrow[loop above,"\sigma"] \arrow[loop below,"Id"] & (2,2) \arrow[loop above,"\sigma"] \arrow[loop below,"Id"] & (3,3) \arrow[loop above,"\sigma"] \arrow[loop below,"Id"] & (2,1) \arrow[loop below,"Id"] & (3,1) \arrow[loop below,"Id"] & (3,2) \arrow[loop below,"Id"]
\end{tikzcd}
\end{figure}

The above diagram is a graphical representation of the situation we face in our previous example. We have a set $B$ of functions $\underline{2}\to I$ neatly represented by ordered pairs, and the symmetric group $\mathfrak{S}_2=\{Id,\sigma\}$ acting on it by means of coordinate permutations. We need to take a `smarter' version of the usual quotient $B/\mathfrak{S}_2$, which preserves the difference between $Id$ and $\sigma$.

Denote by $B//G$ the \emph{weak quotient}, as the category made up of points in $B$ and (iso)morphisms $a\to b$ whenever $ga=b$ for some $g\in G$. Connected components are isomorphic, and a quick argument shows that their automorphism groups are isomorphic too, up to a change in identity; it is reasonable then to treat these components as `objects' with `symmetries'.

In the case above, with $B//\mathfrak{S}_2$, the outer pairs have trivial automorphism groups, but the symmetric pairs $(1,1), (2,2)$ and $(3,3)$ do not! In other words, we have a \emph{quasi}-equivalence relation, where the elements mentioned are equivalent to themselves \emph{in more than one way}. This is the detail that makes the action of the symmetry group into a non-free action, and messes up the cardinality of our fibres; what we really want is to make $E$ into a groupoid too, and find a nice \emph{functor} $E\to B//\mathfrak{S}_2$.

% !TEX root=main.tex

\pagebreak

\section{Note 4}

\subsection{Elementary symmetric functors}

Here we construct a purely combinatorial description of the usual elementary symmetric polynomials, extended to the functor version.

\begin{exercise}
The $k$-th elementary symmetric polynomial functor in $n$ variables $X_1,\dots,X_n$ is given by $I\overset{s}{\gets} E\overset{p}{\to} B$, with sets
\begin{itemize}
\item $I=\{1,2,\dots,n\}$,
\item $B=\{k$-subsets of $I\} = \{[k] \to I\}$, and
\item $E=\{$pointed $k$-subsets of $I\} = \{[1] \to [k] \to I\}$
\end{itemize}
and functions
\begin{itemize}
\item $s:E\to I$ returning the chosen point, and
\item $p:E\to B$ sending a pointed $k$-subset to its unpointed version.
\end{itemize}
\end{exercise}
\begin{solution}
$e_k$ has $\binom{n}{k}$ terms, and $|B|=\binom{n}{k}$. Every monomial has $k$ distinct variables of degree one, and given any $b\in B$, we have that $p\inv(b)$ is a fibre with a total of $k$ pointed $k$-subsets of $I$, producing $k$ distinct variables of degree one. (... a bit more detail ...)

+ show symmetry in combinatorial way
\end{solution}

\begin{solution}
% \emph{REAL SOLUTION.} Fix a permutation $\sigma:I\to I$. We define a natural isomorphism

% \begin{figure}[!ht]
% \centering
% \begin{tikzcd}
% \catset/I \arrow[r, bend left=50, "p_*s^*"{name=U, above}]
% \arrow[r, bend right=50, "p_*s^*\sigma^*"'{name=D}]
% & \catset/B
% \arrow[Leftrightarrow, from=U, to=D,"\eta"].
% \end{tikzcd}
% \end{figure}

%Given any $(X \overset{f}{\to} I)\in\catset/I$, we have to define a bijection $\eta_X:p_*s^*(X)\to p_*s^*\sigma^*(X)$; that is, a bijection of fibres over $B$. This is best worked out in terms of trees! We will care somehow about \emph{restoring} the action of our chosen permutation, so we involve $\sigma\inv$.

Fix a permutation $\sigma:I\to I$, and let $P=p_*s^*$ be the original polynomial, and $Q=p_*s^*\sigma^*$ its permuted version.

%Fix a $b\in B$, and consider $(P(X))_b=\prod_{e\in E_b} X_{s(e)}$; by virtue of $\sigma$ being a bijection, we can substitute $e$ by $\sigma(e)$. We get $\prod_{\sigma(e)\in E_b} X_{s(\sigma(e))}$...

We claim that an isomorphism $\sigma:I\to I$ induces an isomorphism $\tau:B\to B$ such that $\sigma(s(E_b))=s(E_{\tau(b)})$. Recall that $b\in B$ is given by a function $[k]\overset{b}{\to} I$, and let $\tau(b)=[k]\overset{b}{\to} I\overset{\sigma}{\to} I$. Also, $E_b$ is the set of all functions of the form $[1]\overset{c}{\to}[k]\overset{b}{\to}I$. Therefore
\begin{align*}
	\sigma(s(E_b)) &= \sigma(s(\{[1]\overset{c}{\to}[k]\overset{b}{\to} I\}))\\
    	&= \sigma(\{b\circ c(1)\})\\
        &= \{\sigma\circ b\circ c(1)\}\\
		&= \{\tau\circ c(1)\}\\
        &= s(\{[1]\overset{c}{\to}[k]\overset{\tau}{\to} I\})\\
        &= s(E_{\tau(b)}),
\end{align*}
as desired.

Hence we have

\begin{align*}
p_*s^*\sigma^*(X) &= \sum_{b\in B}\prod_{e\in E_b} X_{\sigma(s(e))}\\
    &\simeq \sum_{b\in B}\prod_{e\in E_{\tau(b)}} X_{s(e)}\\
    &\simeq \sum_{b\in \tau(B)}\prod_{e\in E_b} X_{s(e)}\\
    &\simeq \sum_{b\in B}\prod_{e\in E_b} X_{s(e)} = p_*s^*(X).
\end{align*}

For naturality, given any function $X\overset{g}{\to} Y$, we require $\eta_Y\circ P(g) = G(g)\circ \eta_X$ (...)
\end{solution}

So, the overview is as follows: first, we find that a permutation of the indexing set $I$ induces a well-behaved permutation of polynomial terms, that is, of $B$. We use these two isomorphisms $\sigma$ and $\tau$ to manipulate the explicit formula of $P'$ and retrieve the original form of $P$. Naturality presents no trouble.

BNOOOOOOOOOOOOOOOOOOOOOOOOOOOOOOOOOO!!!!!!!!!!! This was for the homogeneous ones, BUT IT DIDNT WORK WITH PURE SETS.... maybe we can save it once we delve into groupoids?

\begin{proposition}
The $k$-th elementary symmetric polynomial functor in $n$ variables $X_1,\dots,X_n$ is given by $I\overset{s}{\gets} E\overset{p}{\to} B$, with sets $I=\{1,2,\dots,n\}$, $B=\binom{I}{k}$, $E=\binom{I}{k'}$, maps $s:E\to I$ returning the pointed element and $p:E\to B$ forgetting the pointed element.
\end{proposition}
\begin{proof}
Our claim is that $p_*s^*\sigma^* \simeq p_*s^*$ for any given permutation $\sigma:I\to I$.

First note that for a given $b\in B$, we have $\sigma(s(E_b))=s(E_{\sigma(b)})$. Hence

\begin{align*}
	p_*s^*\sigma^*(X) &= \sum_{b\in B}\prod_{e\in E_b} X_{\sigma(s(e))}\\
		&\simeq \sum_{b\in B}\prod_{e\in E_{\sigma(b)}} X_{s(e)}\\
        &\simeq \sum_{b\in \sigma(B)}\prod_{e\in E_b} X_{s(e)} = p_*s^*(X).
\end{align*}
\end{proof}

The proof above is quick and dirty; we would like to show the natural isomorphism $p_*s^*\sigma^* \simeq p_*s^*$ without explicitly dealing with sums and products, that is, on a wholly functorial level.

To do that, we observe that the equality $\sigma(s(E_b))=s(E_{\sigma(b)})$ suggests an action on both $E$ and $B$ induced by the $I$-permutation $\sigma$.

We abuse notation by considering $k$ and $n$ to be the sets $[k]=\{1,2,\dots,k\}$ and $[n]=\{1,2,\dots,n\}$ respectively, in the appropriate contexts. By letting $I=[n]$, recall that $B$ can be written as $B=\binom{[n]}{k}=\{k\mono n\}$ and $E$ as $E=\binom{[n]}{k'}=\{1\to k\mono n\}$. Notice that $E,B\subset Set/I$, so action of $\sigma$ on $E$ and $B$ can be written as an endofunctor $\sigma_+:Set/I\to Set/I$, given by
\begin{align*}
(X\to I) \overset{\sigma_+}{\mapsto} (X\to I\overset{\sigma}{\to}).
\end{align*}

\begin{figure}[!ht]
\centering
\begin{tikzcd}
& I \arrow[d,"\sigma"] & \arrow[l,"s"'] E \arrow[d,"\sigma_+"] \arrow[r,"p"] & B \arrow[d,"\sigma_+"] \arrow[r,"t"] & 1\\
X\arrow[r] & I & \arrow[l,"s"] E \arrow[r,"p"'] & B \arrow[ur,"t"']
\end{tikzcd}
\caption{Complete diagram representing the situation for symmetric polynomials.}\label{fig:sympol}
\end{figure}

It is crucial to work with the complete description of a polynomial functor; we have to include the final functor $t_!$ completing the diagram $I\overset{s}{\gets} E\overset{p}{\to} B\overset{t}{\to} 1$. While $t$ is indeed trivial, it plays an important role in our proof.

\begin{lemma}\label{lemma:thelps}
For any isomorphism $\tau:B\to B$ and symmetric polynomial functor $t_!p_*s^*$, there is a natural isomorphism of functors $Q=t_!\tau^*p_*s^* \simeq t_!p_*s^*=P$.
\end{lemma}
\begin{proof}
It is trivially given by $P(X)\to 1 \simeq 1 \gets Q(Y)$.
\end{proof}

The whole proof can be recited out loud by chasing the diagram in Fig. \ref{fig:sympol} along the paths in each line.

\begin{proof}
Succinctly,
\begin{align}
Q = t_!p_*s^*\sigma^* &\simeq t_!p_*\sigma_+^*s^*\\
	&\simeq t_!\sigma_+^*p_*s^*\\
    &\simeq t_!p_*s^* = P.
\end{align}

The first isomorphism is given by commutativity of the leftmost pullback square. The next isomorphism is given by Beck-Chevalley, and the final one depends on our Lemma above.

\end{proof}

\subsection{Complete homogeneous symmetric polynomial functors}



\subsection{Fundamental theorem}

\begin{theorem}[Fundamental Theorem of Symmetric Polynomial Functors]
Given a symmetric polynomial functor $P$ over a category $\mathcal{C}$ with products and sums,
\end{theorem}


\pagebreak
\printbibliography

\end{document}
